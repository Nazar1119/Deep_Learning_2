\documentclass{beamer}
\usepackage[export]{adjustbox}
\usepackage{booktabs}
\usepackage[utf8]{inputenc}
\usepackage{pifont}
\usecolortheme{default}
\title{Conversational Image Understanding with vision models}
\author{Arsenii Milenchuk, Dariia Shevchuk, Nazarii Tymochko, Oleksandr Petsa}
\institute{KKUI / FEI / TUKE}
\date{2025}

\begin{document}

\frame{\titlepage}


% --- define task ---
\begin{frame}
% review of leterature
% uderstand of vlm architecture
% make an evaluation
%
% demo of chat assistant are already complete
\begin{itemize}
\frametitle{Definícia úlohy}
\item Prehľad literatúry - \ding{52} 
\item Pochopenie architektúry - \ding{52}
\item Rozhranie chatu na nadväzovanie konverzácií s large language modelom - \ding{52} 
    \textit{už bol ako pet-projekt v minulosti}
\item Vyhodnotenie rôznych veľkých multimodálnych modelov - \ding{52} 
\item Pridávanie nástrojov pre použitie multimodalnym modelom - \ding{56}
\item Dokumentácia - \ding{56}
\end{itemize}
\end{frame}
% -------------------



% --- define dataset info --- 
\begin{frame}
\frametitle{Dataset}
\begin{block}{Kaggle}
\begin{itemize}
    \item Visual Question Answering-Computer Vision \& NLP
\end{itemize}
\end{block}

\begin{block}{Hodnotenie}
\begin{itemize}
    \item BertScore
\end{itemize}
\end{block}

Zahŕňa pochopenie obsahu obrázka a jeho koreláciu s kontextom položenej otázky. 
Keďže musíme porovnať sémantiku informácií prítomných v oboch modalitách – v obrázku 
a v otázke v prirodzenom jazyku, ktorá s ním súvisí – VQA zahŕňa širokú škálu čiastkových 
problémov v CV aj NLP (ako je detekcia a rozpoznávanie objektov, klasifikácia scén, 
počítanie atď.). Preto sa považuje za úlohu kompletnú s využitím umelej inteligencie.
\end{frame}
% --------------------------- 


% --- review of results ---
\begin{frame}
\frametitle{Precision and Reccal by BertScore}
\includegraphics[width=0.9\textwidth,center]{../imgs/precision_reccal_graph.png}
\end{frame}
\begin{frame}
\frametitle{F1 by BertScore and Compute Time}
\includegraphics[width=0.9\textwidth, center]{../imgs/f1_comp_time_graph.png}
\end{frame}
% -------------------------


% --- best model architecture and usage info ---
    % --- llava ---
    \begin{frame}
    \frametitle{LLaVA Architecture}
    LLaVa spája vopred natrénovaný vizuálny enkodér CLIP ViT-L/14 a 
    large language model Vicuna pomocou jednoduchej projekčnej matice.
    \includegraphics[width=0.9\textwidth, center]{../imgs/llava_arch.png}
    \end{frame}
    % -------------
    % --- minicpm-v ---
    \begin{frame}
    \frametitle{Minicpm-v Architecture}
    Model je postavený na modeloch SigLip-400M a Qwen2-7B s celkovým počtom parametrov 8B. 
    Vykazuje výrazné zlepšenie výkonu oproti MiniCPM-Llama3-V 2.5 a prináša nové funkcie 
    pre pochopenie viacerých obrazov a videa.
    \includegraphics[width=0.9\textwidth, center]{../imgs/minicpm-v_arch.png}
    \end{frame}
    % -----------------
    % --- qwen3-vl ---
    \begin{frame}
    \frametitle{Qwen3-vl Architecture}
    \begin {itemize}
        \item Interleaved-MRoPE
        \item DeepStack
        \item Zarovnanie textu a časovej pečiatky
    \end{itemize}
    \includegraphics[width=0.9\textwidth, center]{../imgs/qwen3-vl_arch.jpg}
    \end{frame}
    % ----------------
% ----------------------------------------------

% --- future work ---
\begin{frame}
Ďakujem za pozornosť
\end{frame}
% -------------------

\end{document}


